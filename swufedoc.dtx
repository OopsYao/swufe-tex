% \iffalse
%<*driver>
\ProvidesFile{swufedoc.dtx}
%</driver>
%<class>\NeedsTeXFormat{LaTeX2e}[1999/12/01]
%<class>\ProvidesClass{swufedoc}
[2021/03/14 1.0.0-alpha.0 SwufeThesis Doc Template]
%
%<*driver>
\documentclass{swufedoc}
\begin{document}
  \DocInput{\jobname.dtx}
\end{document}
%</driver>
% \fi
% \title{\cls{swufedoc}文档类}
% \maketitle
% \begin{abstract}
%   \cls{swufedoc}是\pkg{swufethesis}撰写文档时内部使用的文档类。
% \end{abstract}
% \tableofcontents
% \section{简介}
% \cls{swufedoc}用于\pkg{swufethesis}文档编写,主要进行了一些格式上的设定。
% \DescribeMacro{\SwufeThesis}
% 同时提供有宏\cs{SwufeThesis}可以打印这个项目的标识\SwufeThesis{}。
% \subsection{颜色}
% \cls{swufedoc}提供了如\autoref{tab:colors}所示的一些颜色。
% \newcommand\colorentry[2]{%
%   |#1| & \color{#1}|#2| \\
% }
% \begin{table}[htb]
% \centering
% \begin{tabular}{cc}
%   \toprule
%   颜色名 & 代码 \\
%   \midrule
%   \colorentry{swufe0}{0080FF}
%   \colorentry{swufe1}{71A9F7}
%   \colorentry{swufe2}{0582CA}
%   \colorentry{swufe3}{83BE6C}
%   \colorentry{swufe4}{C8CED9}
%   \colorentry{swufe5}{FAFAFD}
% \bottomrule
% \end{tabular}
% \caption{提供的颜色}
% \label{tab:colors}
% \end{table}
%
% \subsection{代码环境}
% \DescribeEnv{shell}
% \DescribeEnv{latex}
% 提供了\env{shell}与\env{latex}分别作为Shell和\LaTeX{}的代码环境。
% \section{实现细节}
% \subsection{装载类与宏包}
% \cls{swufedoc}基于\textsf{l3doc}文档类,
% 目前的问题是\env{macrocode}环境中的中文字体不遵循\cs{setCJKmonofont}的设定。
% 现在只能尽量减少\env{macrocode}中的中文注释。
% 如果使用\cls{ltxdoc}文档类,甚至连\env{verbatim}环境也会出现上面的问题。
%
% 值得一提的是随\pkg{ctex}宏集一起发行的\cls{ctxdoc}文档类,
% 该类中\env{macrocode}环境的中文字体设定表现正常,但是并没有正式文档(或许是旨在
% 对内使用?),从而也不方便进行更进一步的定制。
% \footnote{事实上\cls{l3doc}也是实验性的,也没有正式发布文档,但提供了 |dtx|文档可供编译。}
%    \begin{macrocode}
\LoadClass{l3doc}
\RequirePackage[fontset=none]{ctex}
\RequirePackage{
  fontspec,
  listings,
  xcolor,
  hyperref,
  booktabs,
  caption,
  etoolbox,
}
%    \end{macrocode}
% \subsection{文档信息相关}
% 提供一些关于描述\SwufeThesis{}本身的命令。
% \begin{macro}{\SwufeThesis}
% \cs{SwufeThesis}会打印“\SwufeThesis”,可以作为本项目的标识。
%    \begin{macrocode}
\def\SwufeThesis{\textsc{Swufe\-Thesis}}
%    \end{macrocode}
% \end{macro}
% \subsection{字体设定}
% 中文使用思源字体系列,西文使用\TeX\ Gyre系列中的Pagella和Heros。
% 如果没有安装思源字体,则会使用\pkg{fandol}字体代替。
%    \begin{macrocode}
\newcommand{\swufe@set@font@pagella}{%
  \setmainfont{texgyrepagella}[
    Extension = .otf,
    UprightFont = *-regular,
    BoldFont = *-bold,
    ItalicFont = *-italic,
    BoldItalicFont = *-bolditalic,
  ]%
  \setsansfont{texgyreheros}[
    Extension = .otf,
    UprightFont = *-regular,
    BoldFont = *-bold,
    ItalicFont = *-italic,
    BoldItalicFont = *-bolditalic,
  ]%
  \setmonofont{inconsolata}
}
\newcommand{\swufe@set@cjk@noto}{
  \setCJKmainfont{Noto Serif CJK SC}[
    UprightFont = * Light,
    BoldFont = * Bold,
    ItalicFont = FandolKai-Regular,
    ItalicFeatures = {Extension = .otf},
  ]%
  \setCJKsansfont{Noto Sans CJK SC}[
    BoldFont = * Medium,
  ]%
  \setCJKmonofont{Noto Sans Mono CJK SC}%
}
\newcommand{\swufe@set@cjk@fandol}{%
  \setCJKmainfont{FandolSong}[
    Extension = .otf,
    UprightFont = *-Regular,
    BoldFont = *-Bold,
    ItalicFont = FandolKai-Regular,
  ]%
  \setCJKsansfont{FandolHei}[
    Extension = .otf,
    UprightFont = *-Regular,
    BoldFont = *-Bold,
  ]%
  \setCJKmonofont{FandolFang}[
    Extension = .otf,
    UprightFont = *-Regular,
  ]%
}
\swufe@set@font@pagella
\IfFontExistsTF{Noto Serif CJK SC}{%
  \IfFontExistsTF{Noto Sans CJK SC}{%
    \swufe@set@cjk@noto
  }{\swufe@set@cjk@fandol}%
}{\swufe@set@cjk@fandol}%
%    \end{macrocode}
% \subsection{颜色设定}
% 设定有主题色西财蓝,并增加了其他的看上去比较协调的颜色。
%    \begin{macrocode}
\definecolor{swufe0}{HTML}{0080FF}
\definecolor{swufe1}{HTML}{71A9F7}
\definecolor{swufe2}{HTML}{0582CA}
\definecolor{swufe3}{HTML}{83BE6C}
\definecolor{swufe4}{HTML}{C8CED9}
\definecolor{swufe5}{HTML}{FAFAFD}
\colorlet{SwufeBlue}{swufe0}
%    \end{macrocode}
% \subsection{代码环境}
% 通过\pkg{listings}定义一些代码环境\env{latex}、\env{shell}。
%    \begin{macrocode}
\lstdefinestyle{lstStyleBase}{%
  basicstyle=\small\ttfamily,
  linewidth=\linewidth,
  gobble=2,
  tabsize=2,
  showstringspaces=false,
  breaklines=true,
  breakatwhitespace=false,
  breakindent=0pt,
  columns=flexible,
  frame=l,
  framerule=1pt,
  rulesep=1pt,
  backgroundcolor=\color{swufe5},
  stringstyle=\color{swufe3},
  keywordstyle=\bfseries\color{swufe1},
  commentstyle=\slshape\color{swufe4},
  rulecolor=\color{swufe0},
}
%    \end{macrocode}
% \begin{environment}{latex}
% \env{latex}环境。
%    \begin{macrocode}
\lstnewenvironment{latex}{\lstset{style=lstStyleBase, language=[LaTeX]TeX}}{}
%    \end{macrocode}
% \end{environment}
% \begin{environment}{shell}
% \env{shell}环境。
%    \begin{macrocode}
\lstnewenvironment{shell}{\lstset{style=lstStyleBase, language=bash}}{}
%    \end{macrocode}
% \end{environment}
% \subsection{超链接样式}
% 通过\pkg{hyperref}进行一些超链接的设定,包括文档内部跳转的链接、
% 外部的URL。
%    \begin{macrocode}
\hypersetup{
  linkcolor = swufe2,
  urlcolor = swufe1,
}
%    \end{macrocode}
%
% \subsection{图表设定}
% 小号的图表字体,无衬线的图表标题。表格字体是通过修改\env{tabular}
% 环境实现的,所以其他地方若以\env{tabular}作其他用途时需要注意。
%    \begin{macrocode}
\AtBeginEnvironment{tabular}{\small}
\captionsetup{
  font = {sf, small},
  labelfont = {bf},
}
%    \end{macrocode}
%
% \subsection{交叉引用}
% 利用\pkg{hyperref}提供的\cs{autoref}进行交叉引用,这里进行一些
% 相应的汉化。但是\cs{autoref}并不支持同时引用多条,或许可以考虑
% \pkg{cleveref}等宏包。
%    \begin{macrocode}
\renewcommand{\tableautorefname}{表}
\renewcommand{\figureautorefname}{图}
\renewcommand{\equationautorefname}{式}
%    \end{macrocode}
